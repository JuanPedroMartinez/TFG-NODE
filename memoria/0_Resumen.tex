\chapter*{Extended Abstract\label{00Abstractsummary}}

\addcontentsline{toc}{chapter}{Extended Abstract}

Nowadays, Git\cite{git} is by far the most widely used modern version control system
in the world. Version control is a system that records changes to a
file or set of files over time so that you can recall specific versions
later. Git is a mature and actively maintained open source project, with
tons of developers behind it. Originated and launched in 2005, by the
creator of the Kernel of Linux systems, Linus Torvalds, has become a
crucial tool for the development teams, bringing them better performance,
flexibility, and security for their projects.

Some of the reasons that make this tool so widely used may
be,~(i)~Teamwork, allowing parallel development on a project with a shared
access.~(ii)~Autonomy, each developer, has a full local copy of the project
and the changes generated on it, which let developers work
individually.~(iii)~Speed, Git need less processing requirements and
management than other version control systems.~(iv)~Tree structure, teams
can have different branches, on which they can make changes without
modifying the principal base code, making the testing easier and making
possible the creation of alternative solutions.~(v)~Scalability and fault
tolerance, open source, multi platform implementations and community
support.

The rise of this technology has been so great that multiple faculties
related to software education integrate it into their teaching, using this
tool and integrating it in the development of the projects to be
accomplished by their students. When this happens, the students create
repositories making the teacher a collaborator of them. In that way, the
teacher has full access to the repository where the students work
progressively, allowing cooperative work in real time.

On the other hand, the teaching staff can later analyze the progression and
work of the students, checking the changes made in the repository, in such
a way that they can observe what changes each student has made, how the
work has been distributed, detect overload of work of some student with
respect the rest, and much more. In this way, the teacher can offer a rating
as fair as possible.

However, for teachers, at time of evaluation, it can be hard to check the
data offered by their student’s repositories, especially when the number of
repositories increases. Imagine a professor who teaches different subjects,
and for each of them the students use different repositories.

Currently, the are some solutions for the problem, such as the stats panels
of Git service providers, such as GitHub\cite{GitHub} or GitLab\cite{GitLab}. However in case of
GitHub, these stats can only be consulted on public repositories, to be
able of check them at private repositories, a premium account is needed
under a monthly subscription. On the other hand, GitLab stats panel is
offered for private repositories for free, but access to the changes made
in the repository cannot be easily visualized together with the stats,
having to open another page for viewing them, becoming somewhat tedious in
the case of have many repositories.

These services providers, GitHub and GitLab, also offer payment plans
dedicated to education, these plans include multiple advantages, being more
interesting in the case of GitHub, offering a tool designed exclusively for
education “GitHub ClassRoom”, this tool allows the creation of virtual
classrooms, where teachers can create templates of projects and share for
the students, create homeworks for the students which can be self-assessed,
assign work teams, in addition of have full control of the changed made by
the students, with multiple statistics and facilities for students
follow-up, allowing a fast and direct interaction between teachers and
students.

As we can see, the solutions offered are not entirely effective, the one
that probably best suits the problem posed is the GitHub Education plan,
nevertheless, although many subsidies and grants can be requested, the
service entails a cost for the institution and also the necessary
management to form the agreement between the institution and GitHub must be
taken in account. In case of not being able to apply this solution, it
would have to be the teaching staff, who would have to either get a premium
account or contract the GitHub services on their own. Or manually inspect
all the repositories.

For the reasons mentioned previously, we want to develop a software for the
computer science faculty of the University of Murcia, that allows teachers
to manage the information of their students' repositories, obtaining
statistics about them, simplifying their task. This would be a first
implementation, with lots of room for improvement, with the possibility of
implementing new features.

In addition, it is intended to offer a useful implementation for deployment
in a cloud service, in such a way that the services are based on
containers, which can be launched in a structure of cloud nodes, thus
allowing their control and scaling.

In order to continue, we must first understand the Git jargon, and some
relevant definitions related to the developed project. As a glossary, we
see the following definitions:

\begin{itemize}
\item Repository: A Git repository tracks and saves the history of all
  changes made to the files in a Git project. It saves this data in a
  directory called ``\texttt{.git}'', also known as the
  repository folder. Git uses a version control system to track all
  changes made to the project and save them in the repository.

\item Commit: In version control systems, a commit is an operation which
  sends the latest changes of the source code to the repository, making
  these changes part of the head revision of the repository. Commits in
  version control systems are kept in the repository indefinitely. Thus,
  when other users do an update or a checkout from the repository, they
  will receive the latest committed version.

\item Branch: is the duplication of an object under version control (such
  as a source code file or a directory tree). Each object can thereafter be
  modified separately and in parallel so that the objects become different.
  In this context the objects are called branches. The users of the version
  control system can branch any branch. Branches allow you to develop
  features, fix bugs, or safely experiment with new ideas in a contained
  area of your repository.

\item REST API: A RESTful API is an architectural style for an application
  program interface (API) that uses HTTP requests to access and use data.
  That data can be used to GET, PUT, POST and DELETE data types, which
  refers to the reading, updating, creating, and deleting of operations
  concerning resources.
\end{itemize}

Once these concepts are understood, we can begin by explaining the software
requirements. The proposed solution is based on the development of a web
application that allows the management of teacher’s repositories. To do
this, each teacher will make a registration in the application, in which
they can later add their subjects, being able to filter their repositories
by matching strings and manually add repositories to their subjects. For
the registration process, each teacher must have or create an
authentication token obtained on the official GitHub page,necessary for the
developed application to access the private repositories of teachers. The
tool will never make any changes to the repositories or add any information
to git, it will only need read permissions.

Once the registration is complete, you can access the application and
view all the repositories in which you participate as a collaborator
on the main page. In addition, a search bar is offered, for filtering
the repositories that match a string of characters and filtering by
previously created subjects. You can click on any of those displayed
to access the repository statistics. Obtaining information about the
collaborators of the repository, the total number of commits made by
each collaborator, the temporal distribution of the commits, that is,
how many commits have been made in each month of the course, and the
content of the changes made in the project on every commit. With this
information, the teacher can evaluate the performance of the students
during the development of the assigned work.

Turning now to the implementation of the software, there were
different alternatives, as for the development of the web server, the
possibility of using Python\cite{Python,PhytonGithub} with the Flask\cite{Flask} framework, or using NodeJS\cite{NodeJS,nodeDoc,CodeShack}
based on the Express\cite{ExpressJS,GitBook,DevDocs} framework. Finally, the decision was to use NodeJS given the good integration of the JSON\cite{json} format used by the Github
API through which we retrieve the data from the repositories, in
addition to the many possibilities that the Express Framework offer
for the development of simple API REST.

Regarding the persistence of the data, necessary in this case for the
data of both the users and the subjects created, a MySQL\cite{MySQL}
implementation is used, since the data to be persisted perfectly
adapts to the relational model, and NodeJS offers very good libraries
for connection and integration with the database from the server.

Finally, for design reasons, a second server has been implemented,
also developed in NodeJS Express, which is in charge of carrying out
all the communication with the Github servers, that is, it will be in
charge of obtaining all the necessary information from the Github
services. In this way, we have:

\begin{itemize}
\item The first server, which implements the entire web service, is
  intended only to receive and respond to web requests from users.
  This server will also be in charge of communicating with the
  database and managing it, carrying out registration queries and user
  creation, as well as managing the subjects. If data from the Github
  Rest API is necessary, this data is requested from the second
  server.

\item The second server implements a service as a REST API in charge
  of attending to the requests of the first server and
  intercommunicating with the GitHub API to obtain the requested data.
  Therefore, it acts as a proxy for the GitHub service.
\end{itemize}

The reasons for implementing a second server are justified by
assigning only the web server to users and not overloading it with
requests to another external service, thus reducing its workload and
improving response times.

On the other hand, for the communication of the two servers, the
second server is implemented as a REST API, where it receives HTTP
requests, in this case it only accepts Get requests, the information
about what data you want to recover, is received through the URL
parameters. Once the request is received and processed, the second
server communicates with the GIthub API, obtains the corresponding
data and responds back to the first server.

As a second objective for the project, we want to give support to the
services developed in order to facilitate deployment in a cloud
service.

To do this, all services must offer a container based implementation,
in this case using Docker\cite{Docker} container technology, currently, one of
the most relevant technologies for this task.

The structure of the application is divided into~3 containers, a
dedicated container for the database, which must configure a volume to
give persistence to the data in the container, and two different
containers for each of the servers implemented in NodeJS.

To make the deployment of these containers easier, a Docker Compose
configuration file is provided, which allows us to automatically
launch the~3 services, with the specific configurations for each one.
It is also prepared to configure an internal subnet used to resolve
the communications between the services, that is, the communications
between the main server and the MySQL database and the communications
between the first and second servers for obtaining Git data.

Regarding the work methodology followed, multiple meetings have been
held with the tutor Diego Sevilla, in which the idea of the project
has been raised and evolved.

%nueva conclusión a ver si es mas apropiada.

In addition, for the development of the work, a private Git repository
was shared, where the tutor could check the progress and see the
entire project to help me with the queries.

To conclude, comment that the development of the project has entailed
a lot of work, but it has been a very rewarding experience, and key to
consolidating technologies that are currently on rise, such as the
development of microservices and the collaboration between them to
provide solutions for multiple problems, and its subsequent deployment
on cloud architectures.

In my opinion, the project after the development and implementation is
a fully usable tool that meets the requirements established for the
project. Nevertheless, the tool can be improved in many ways, among
which we highlight the integration with GitLab services, another of
the great known as Git service provider and which even being smaller
than GitHub, It also has a great community, or the implementation of
new features such as the ability to download repositories, make
commits, etc. On the other hand, it could be used as a starting point
for the development of a classroom tool, where projects can be created
and distributed, also offering tools for correcting student projects,
etc.




%\hilight{Lo siguiente no me parece bien en el resumen, pero sí que
 % habría que añadir algo así como una conclusión de que finalmente se
 % ha creado lo que se quería, etc. Ya he realizado los cambios.}



% In the first meeting, the development of a tool that tries to solve
% the problem of analyzing student repositories is planned. Different
% forms of implementation and technologies to be used are also proposed.

% Subsequently, after a period of research on how to retrieve statistics
% from the

% Git servers, in this case Github, a second meeting took place, where
% it is concisely specified what the architecture of the application
% will be and the technologies that will be used.

% After a period of work, other meetings have been held where the
% progress made has been reviewed, new application requirements have
% been proposed and some aspects have been improved.

% Successively, there have been multiple consultations throughout the
% development of the project, both by email and by shorter face-to-face
% meetings.

% Finally, it should be noted that, as it could not be missing, a
% repository has been used

% Private GitHub to share work with tutor and version control.

% To conclude, comment that the development of the project has entailed
% a lot of work, but it has been a very rewarding experience, and key to
% consolidating technologies that are currently on rise, such as the
% development of microservices and the collaboration between them to
% provide solutions for multiple problems, and its subsequent deployment
% on cloud architectures.

\chapter*{Resumen\label{00summary}}

\addcontentsline{toc}{chapter}{Resumen}

Hoy en día, Git\cite{EDteam,CodicesSoftware,Uqbar,Sivsa} es, con diferencia el sistema de control de versiones
moderno más utilizado en todo el mundo, es un proyecto de código
abierto maduro y con un mantenimiento activo con infinidad de
desarrolladores que lo respaldan. Originado y lanzado en 2005, por el
famoso creador del núcleo del sistema operativo de linux, Linus
Torvalds, se ha convertido en una herramienta casi indispensable en
los equipos de desarrollo software, ofreciendo un mejor rendimiento,
flexibilidad y seguridad al equipo de desarrollo. Entre los múltiples
motivos por los que Git se ha convertido en uno de los sistemas de
control de versiones más utilizados, encontramos trabajo en equipo en
tiempo real con acceso compartido, autonomía, velocidad de las
operaciones, estructura en árbol contando con diferentes ramas para
facilitar pruebas y vías alternativas en el desarrollo, escalabilidad,
código libre y apoyo de la comunidad.

Tanto ha sido el auge de esta tecnología, que múltiples facultades
relacionadas con el software, lo integran en sus docencias, formando
parte de la propia evaluación. Cuando esto ocurre, los alumnos forman
sus repositorios a los cuales el profesor tiene acceso, de esta forma
el o los alumnos trabajan progresivamente, permitiéndoles el trabajo
cooperativo en tiempo real.

Por otro lado, el profesorado, puede analizar posteriormente el
trabajo de los alumnos comprobando los cambios realizados en el
repositorio, de tal forma que puede observar que cambios ha hecho cada
alumno, cómo se han repartido el trabajo, sobrecarga de algún alumno
respecto al resto y de esta forma ofrecer una calificación lo más
justa posible.

Sin embargo, de cara al profesorado, puede ser complicado comprobar
los datos ofrecidos por los repositorios de sus alumnos, especialmente
cuando el número de alumnos se incrementa.

Actualmente, existen algunas soluciones existentes para el problema,
como pueden ser los paneles de estadísticas de proveedores de
servicios Git como Github o Gitlab. Sin embargo, estas alternativas,
posteriormente analizadas en los posteriores capítulos, en algunos
casos requieren de un pago o bien por parte de la institución o por
parte del profesorado, y en otros casos, las alternativas gratuitas no
terminan de ser fácilmente aplicables para el contexto en el que se
quieren usar, siendo una tarea tediosa cuando el número de
repositorios a analizar es grande.

Estos proveedores de servicios, también ofrecen planes de pago
dedicados a la educación, estos planes, incluyen múltiples ventajas,
siendo más interesantes en el caso de GitHub, ofreciendo una
herramienta dedicada a la creación de aulas virtuales, donde mediante
repositorios, se puede compartir plantillas creadas por el profesor,
crear tareas, tanto individuales como grupales, creación de tareas
autoevaluables, asignar equipos de trabajo, además de ofrecer un
control total al profesorado de todos los cambios realizados, con
múltiples estadísticas y facilidades para el control de los alumnos,
permitiendo una interacción rápida y directa.

Por tanto, se quiere desarrollar un software, desarrollado en la
facultad de informática de la universidad de Murcia, que permita a los
profesores, gestionar la información de los repositorios de sus
alumnos, obteniendo estadísticas acerca de ellos, simplificando su
tarea, y iniciando un proyecto con amplio margen de mejora, con la
posibilidad de implementar nuevas funcionalidades.

Además, se pretende ofrecer una implementación útil para el despliegue
en un servicio cloud, de tal forma que los servicios, estén basados en
contenedores, que puedan ser lanzados en una estructura de nodos
cloud, permitiendo así su control y escalado.

En cuanto los requisitos del proyecto software, la solución propuesta
está basada en el desarrollo de una aplicación web que permita la
gestión de los repositorios de los profesores. Para ello, cada
profesor realizará un registro en la aplicación, en la cual
posteriormente podrá registrar sus asignaturas, pudiendo filtrar sus
repositorios por cadenas de coincidencia y añadir repositorios de
forma manual a sus asignaturas. Para su registro, cada profesor debe
contar con un token de autenticación obtenido en la página oficial de
github, con el que se dará acceso a la aplicación a los repositorios
privados.

Una vez completado el registro, se puede acceder a la aplicación y
visualizar en la página principal todos los repositorios en los que se
participa como colaborador. Además, se ofrece una barra de búsqueda,
para el filtrado de los repositorios que coincidan con una cadena de
caracteres y filtrado por asignaturas creadas previamente. Sobre
cualquier de los que se visualizan se puede hacer click para acceder a
las estadísticas del repositorio. Obteniendo información acerca de los
colaboradores del repositorio, el número total de commits realizados
por cada colaborador, la distribución temporal de los commits, es
decir, cuantos commits se han realizado en cada mes del curso, y el
contenido de los cambios realizados en el proyecto en cada commit. Con
esta información el profesor puede evaluar el desempeño de los alumnos
durante el desarrollo del trabajo asignado.

Para la implementación del software, se utilizarán tecnologías NodeJS
para el desarrollo del backend dando soporte al mismo mediante una
base de datos MySQL. Por otra parte, en cuanto a la contenerización de
la aplicación, se utilizará Docker para la generación y posterior
despliegue\cite{InitMySQLDocker,MYSQL-DOckerHUB,Docean-docker,Docean-dockerCompose} de los contenedores mediante Docker-Compose.

Para concluir, comentar que el desarrollo del proyecto, ha conllevado
bastante trabajo, pero ha sido una experiencia muy gratificante, y
clave para profundizar en tecnologías en auge en plena actualidad,
como son el desarrollo de los microservicios y la colaboración entre
ellos para dar solución a múltiples problemas , y su posterior
despliegue sobre arquitecturas cloud.
%%% Local variables:
%%% TeX-master: "memoria.tex"
%%% coding: utf-8
%%% ispell-local-dictionary: "english"
%%% TeX-parse-self: t
%%% TeX-auto-save: t
%%% fill-column: 75
