\chapter*{Declaración firmada sobre originalidad del
  trabajo\label{cap:orginalidad}}
\addcontentsline{toc}{chapter}{Declaración firmada sobre originalidad del
  trabajo}

De acuerdo al Reglamento que regula los trabajos fin de grado en la
Universidad de Murcia, DECLARO que asumo la originalidad del trabajo
presentado y que todas las fuentes utilizadas han sido debidamente citadas.

\vspace{10em}

\begin{raggedleft}
Murcia, 7 de septiembre de 2021\\

Nombre y firma\\

Firmado Óscar Hernández Navarro\\

%\begin{figure}[H]
%\includegraphics[width=0.3\textwidth,right]{img/sign}
%\end{figure}
\end{raggedleft}

\chapter*{Extended Abstract\label{00Abstractsummary}}
With the arrival of Web 2.0, large companies were in need of expanding their services due to the great information traffic that began to generate, because of this nosql databases were born. Databases whose initial purpose was to provide horizontal scalability to these services and replace the old relational databases.

These databases were implemented by large computer companies in the world and each time they emerged more. They began to be used in areas such as academic research, Machine Learning and other computational sciences due to their flexibility and availability.

They also emerged from the community, and therefore the vast majority are open source and free.

However, twenty years later, the necessary tools have not been created in order to manage these databases as well as the relational, that is why many companies continue to use old databases that have a great support behind and tens of years of improvements.

One of these essential tools is the generation of pseudoameratory data. This generation is necessary in many aspects. From the comparatives between different Nosql databases until the performance tests passing through the handling of the errors.
They are also necessary to verify that the built software is suitable for the selected database and vice versa.

\chapter*{Resumen\label{00summary}}
\addcontentsline{toc}{chapter}{Resumen}

Resumen...

%%% Local variables:
%%% TeX-master: "memoria.tex"
%%% coding: utf-8
%%% ispell-local-dictionary: "spanish"
%%% TeX-parse-self: t
%%% TeX-auto-save: t
%%% fill-column: 75
%%% End:
