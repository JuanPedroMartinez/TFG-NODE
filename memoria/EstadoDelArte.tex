\chapter{Estado del arte\label{02estadoArte}}

\section{Análisis de estadísticas sobre repositorios Git}

Actualmente podemos encontrar multiples vías y herramientas con las que
poder visualizar los datos acerca de los repositorios Git. En este caso,
vamos a analizar las herramientas ofrecidas para el control de
estadísticas por los proveedores de servicio, teniendo en cuenta dos de
los proveedores mas reconocidos, como son
Github\footnote{\url{https://github.com/}.} y
Gitlab\footnote{\url{https://gitlab.com/}.} y subscripciones a planes de
pago, los cuales se adaptan a un entorno educativo, donde profesores y
alumnos cuentan con herramientas para el control y planificación del curso.

Vemos a continuación los aspectos más relevantes de cada una de ellas:

\subsection{Estadísticas de GitHub}

\subsubsection{Estadísticas de GitHub (Insights)}

Desde la propia página web de GitHub, o alternativas como GitHub Desktop o
terminal, accediendo a un repositorio propio o del cual se es colaborador,
en el apartado \emph{Insights}, se ofrece un amplio catálogo de
estadísticas para consultar, contando entre otros, estadísticas sobre
contribuidores, tráfico de datos en el repositorio, información acerca de
los commits realizados, frecuencia de código en el proyecto, ramas, etc.

Es una muy buena herramienta, con una apariencia clara y limpia, con una
interfaz muy intuitiva. Sin embargo, esta característica únicamente es
accesible a repositorios públicos, o mediante la obtención de una cuenta
GitHub Premium, que requiere del pago de cuota mensual.

En cuanto al uso del profesorado, principalmente encontramos la desventaja
de que los repositorios de los alumnos no pueden ser públicos, además del
requerimiento de los profesores de tener una cuenta premium. Por otra
parte, no todos los alumnos usan GitHub como servicio para repositorios
Git, también se ofrece la posibilidad de usar GitLab o otras alternativas,
lo cual haría que esta solución deje de ser válida para estos casos.


\begin{figure}[h!]
  \includegraphics[width=1\textwidth]{GItHub-Insights.png}
  \caption{Ejemplo de la web de estadísticas en la página oficial de
    GitHub.}
  \label{figure:GithubInsights}
\end{figure}

En la figura~\ref{figure:GithubInsights} se muestra un ejemplo de un panel
de estadísticas de los múltiples que ofrece GitHub. Como vemos, tienen una
apariencia limpia y muy clara, con una interfaz sencilla de usar y muy
cómoda. Para el ejemplo se ha tomado el repositorio de "vscode" de
microsoft, como vemos es un repositorio público, lo que permite ver las
estadísticas completas, como ya se ha comentado, en caso de querer
visualizarlas para un repositorio privado al cual tengamos acceso, ya sea
siendo propietarios del mismo o como colaborador, necesitaríamos contar con
una cuenta premium para poder visualizarlas, teniendo que realizar un pago
mensual para ello.

En la parte izquierda, podemos visualizar un menú seleccionable, en el cual
elegimos qué estadísticas queremos visualizar, ofreciendo distintas
posibilidades, siendo para el caso tratado las mas interesantes las
relacionadas con los commits, contribuidores y la frecuencia con la que se
ha programado.

\subsubsection{GitHub Education}

Github Education\footnote{\url{https://education.github.com}.}, de la mano
de GitHub, ofrece múltiples servicios de pago destinados a la
educación. Estos servicios son aplicables a distintas instituciones
educativas y de distintos tamaños. Estas pueden llegar a un convenio con
GitHub, contando con múltiples tipos de ayudas para la contratación de
servicios.

Entre los servicios distinguimos entre herramientas para el profesorado y
herramientas para el alumnado. Ofreciendo diferentes ventajas destinadas a
los requerimientos de cada uno. En este caso, el análisis está más enfocado
a las ventajas que puede obtener el profesorado, a modo de comparativa con
la herramienta desarrollada, cuyo fin es el soporte a los profesores para
llevar el control de los alumnos.

Desde la página web de GitHub\footnote{\url{https://education.github.com/discount_requests/teacher_application}},
encontramos una guía donde se indican claramente los pasos que se han de
seguir para darse de alta como profesor, indicando que el proceso conlleva
alrededor de unos~15 minutos. Una vez realizado el registro, se cuentan con
las siguientes herramientas:

\begin{itemize}
\item Acceso a ``Education Community'', un lugar donde los educadores
  pueden comentar sus ideas acerca de las tendencias de la educación
  tecnológica. Pudiendo así comentar, investigar y aprender nuevas formas
  de comunicación, enseñanza y esquemas de trabajo para sus alumnos.
  Además, se pueden consultar dudas acerca de cualquier problema
  relacionado con el uso de las herramientas ofrecidas, contando con una
  amplia comunidad activa, que da solución a múltiples consultas y
  problemas.
\item Posibilidad de solicitar un ``botín de GitHub'', incluyendo este
  beneficios educativos y material destinado a los estudiantes. En caso de
  ser aprobada la solicitud, se reciben múltiples tutoriales y guías sobre
  el uso de Git, posters, stickers y algunas tarjetas de regalo de
  camisetas de GitHub, canjeables por los alumnos en la web, destinados a
  ofrecer a modo de premio a los estudiantes con mejor desempeño.
\item Acceso a ``GitHub Teams'', permitiendo tener un número ilimitado de
  usuarios y repositorios privados.
\item El software completo de “GitHub classroom”, una aplicación web tanto
  para docentes como alumnos, que permite entre otras,
  \begin{itemize}
  \item La creación de aulas virtuales, donde los alumnos y maestros
    interactúan a lo largo de la duración del curso. Un mismo profesor
    puede crear diferentes aulas virtuales donde se impartirán diferentes
    asignaturas posibles.
  \item Creación de tareas, tanto de manera
    individual como grupales, creándolas algún de los docentes con permiso
    en el aula y consiguiendo un enlace el cual los alumnos usarán para el
    acceso (Figura~\ref{figure:GitHubClassroomTareas}).
  \item Crear plantillas a partir de un repositorio inicial de tal forma
    que el código sea sencillamente distribuible a los alumnos.
  \item Programar tareas con una calificación automática, es decir, una vez
    los alumnos entregan la tarea, siguiendo las indicaciones asignadas a
    la tarea, esta se autoevalúan, teniendo acceso en tiempo real de parte
    de los alumnos de las calificaciones y haciendo más eficiente el
    trabajo del profesor (Figura~\ref{figure:GitHubClassroomTareas})
  \item Control total de los repositorios creados sobre el aula de trabajo,
    permitiendo tanto comunicación con los alumnos, como un amplio catálogo
    de estadísticas y herramientas para la evaluación. Permite también la
    evaluación y corrección de los repositorios de los alumnos de forma
    rápida y sencilla, pudiendo realizar comentarios sobre los commits
    creados y teniendo interacción con los alumnos.
  \end{itemize}
\item Solicitud de servicios cloud, aplicados a la enseñanza de nuevas
  tecnologías, contando con colaboradores de alta influencia en la
  actualidad de los servicios cloud, como lo son Digitalocean o Azure.
\end{itemize}

\begin{figure}[h!]
  \includegraphics[width=\textwidth]{GithubClassromEquipos.png}
  \caption{Panel de creación de tareas tanto individuales como en grupo,
    por parte del/los tutores.}
  \label{figure:GitHubClassroomTareas}
\end{figure}
\begin{figure}[h!]
  \includegraphics[width=\textwidth]{GithubClassroomTareasIndividuales.png}
  \caption{Gestión de tareas grupales y creación de equipos por parte de
    los alumnos, gestionada automáticamente.}
  \label{figure:GitHubClassroomEquipos}
\end{figure}


\subsection{Estadisticas de GitLab}

En el caso del proveedor GitLab, sí se ofrece soporte para la consulta de
estadísticas de los repositorios, permitiendo ver gráficas sobre los
commits realizados, la distribución temporal de los mismos. Además se
ofrecen datos acerca del ``coverage'' del código del repositorio, y
porcentajes de uso en caso de utilizarse más de un lenguaje de programación
sobre el proyecto.

\begin{figure}[h!]
  \includegraphics[width=\textwidth]{GitLab-stats.png}
  \caption{Ejemplo de la web de estadísticas en la página oficial de
    GitLab.}
  \label{figure:GitLabInsights}
\end{figure}

En la figura~\ref{figure:GitLabInsights}, se muestra una captura de la web
de GitLab, y mostrando información relacionada con un repositorio público,
podemos encontrar a la izquierda una sección en la que seleccionar las
diferentes analiticas ofrecidas para el repositorio.

Respecto a la accesibilidad de los datos, en el caso de la web de GitHub,
se muestran de forma más clara y accesible, conteniendo además datos de
mayor relevancia para el propósito que se está tratando y más cantidad de
información. En cuanto el posible uso del profesorado de GitLab para la
consulta de estadísticas, se destaca que el acceso a los repositorios, no
es el más rápido, en mayor instancia en caso de ser un número alto de
repositorios, pudiéndose hacer difícil y tedioso la consulta de todos
ellos.

\subsubsection{GitLab for education}

\textbf{GitLab for
  Education}\footnote{\url{https://about.gitlab.com/solutions/education/}.}:
Al igual que GitHub, GitLab también ofrece servicios destinados a la
educación, en este caso también son servicios de pago y como en el otro
caso, con posibilidad de solicitar múltiples tipos de subvenciones para la
institución donde se quiera aplicar el programa para la educación de
GitLab.

En cuanto a lo ofrecido, los principales objetivos del programa de
educación son:

\begin{itemize}
\item Creación de cuentas para los usuarios de la institución educativa sin
  límites.
\item Colaboración entre profesorado y alumno.
\item Acceso al foro de GitLab con categorías específicas para la educación
  y un amplio catálogo de miembros.
\item Guías para el alumnado sobre las tecnologías ofrecidas
\item Acceso a los eventos ``Hackaton'' donde los alumnos pueden competir
  en jornadas de programación, colaborando con problemas de la comunidad, y
  accediendo a un amplio número de premios para los mejores contribuidores.
\item Ofrece a los alumnos una visión del ciclo de vida DevOps, pudiendo
  aplicar sobre los repositorios integración y entrega continua,
  permitiendo así desarrollar, probar y desplegar el código de forma más
  sencilla. GitLab ofrece herramientas propias con las que se posibilita
  esta tarea y pudiendo realizar las pruebas en remoto, contando con~50.000
  minutos de CI/CD.
\end{itemize}

En cuanto a las diferencias con GitHub education, ofrecen servicios que no
solo son válidos en la educación, ofreciendo distintos planes tanto para
empresas e instituciones de cualquier tipo. El servicio de GitHub está
plenamente enfocado en la educación, y ofrece un control al profesorado más
amplio que GitLab, permitiendo la creación de aula, tareas de distintos
tipos, etc.


\section{Limitaciones de las herramientas existentes y características
  deseadas de la herramienta}

Las herramientas de análisis de repositorios, como hemos visto, en general
requieren una suscripción de pago, excepto en el caso de GitLab en la que
su panel de estadísticas sí se ofrece de manera abierta tanto para
repositorios públicos como privados. Sin embargo, en los paneles de
estadísticas ofrecidos por ambos proveedores, no se muestra de forma rápida
el contenido de los commits, teniendo que acceder previamente a los commits
y visualizarlos uno por uno. Esto es un aspecto bastante negativo ya que
para la correción de los alumnos, es necesario visualizar el interior de
los commits, ya que en muchos casos estos pueden contener simples
modificaciones que realmente no conllevan un gran trabajo por parte de
dicho alumno. Esto indica que únicamente con visualizar el número de
commits de cada alumno y la distribución temporal de la misma no es
suficiente, y es necesario visualizar el contenido de los mismos.
Encontramos por tanto, que el profesorado debe acceder por separado a los
commits y las estadísticas en las soluciones propuestas, esto repetido para
cada uno de sus alumnos.

Por otra parte, sobre los servicios ofrecidos especificamente para las
instituciones, son una gran oportunidad de integrar en la institución la
herramienta de Git, ofreciendo múltiples ventajas tanto para la misma
institución como para sus integrantes. Sin embargo, estas soluciones
requieren de un convenio entre la institución y el proveedor de servicios.
Esto en muchos casos no es sencillo, conllevando múltiples gestiones y
requerimientos, además de tener en cuenta el coste que conlleva
para la institución la integración de dichos servicios.

Como conclusión, tras conocer diferentes alternativas válidas para el
control del trabajo realizado por parte del alumnado, podemos destacar las
características que serían deseables para la herramienta a desarrollar en
este trabajo:

\begin{itemize}
\item Una herramienta libre de pagos, tanto por parte del profesorado como
  de la institución, y ofrecida de forma gratuita para cualquier
  institución que quiera aplicarla.
\item Control de los datos del profesorado, estos datos se manejan de forma
  interna en la institución de la universidad de Murcia, teniendo el
  control total de dichos datos, y no dependiendo de un servicio externo.
\item Herramienta desarrollada por la propia Facultad, con un margen de
  mejora muy amplio y posibilidad de implementar nuevas funcionalidades,
  que se adapten específicamente a las necesidades de la institución,
  pudiendo además extrapolar a distintas facultades para la gestión de
  distintos proyectos.
\item Realizar un proyecto plenamente escalable, mediante el uso de
  contenedores Docker, pudiendo ofrecer un buen rendimiento incluso bajo un
  nivel de requerimiento alto.
\item Un acceso rápido y eficiente a las estadísticas de los repositorios
  de los alumnos, obteniendo la información necesaria con el menor número
  de ``clicks'' posible.
\item Ofrecer una forma de trabajo a los profesores que les permita
  gestionar de manera sencilla un conjunto de asignaturas en las que en
  cada una los alumnos tendrán un repositorio Git.
\end{itemize}

%%% Local variables:
%%% TeX-master: "memoria.tex"
%%% coding: utf-8
%%% ispell-local-dictionary: "spanish"
%%% TeX-parse-self: t
%%% TeX-auto-save: t
%%% fill-column: 75
%%% End:

%  LocalWords:  Scala interoperabilidad metamodelado metamodelo Ecore
%  LocalWords:  Sirius
