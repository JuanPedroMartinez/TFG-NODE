\chapter{Objetivos del proyecto\label{03analisisObjetivos}}

Este proyecto pretende ofrecer una herramienta desarrollada en la facultad de Informática de la Universidad de Murcia, que permita analizar los repositorios de Git en múltiples casos usados por sus alumnos, facilitando la tarea de la corrección, y aportando la información que refleja el trabajo por parte de los alumnos al trabajar en sus repositorios. Pudiendo obtener de dichos datos, la frecuencia de trabajo de los alumnos, casos en los que puede que algún alumno haya tenido un mayor desempeño y carga de trabajo que otro.

Por otra parte, se tiene como objetivo poder llevar la herramienta desarrollada a un servicio real, pudiendo hacer accesible el servicio para la facultad, por lo que el servicio debe contar con una arquitectura que permita su despliegue y escalado.

Como requisitos de la aplicación encontramos:

\begin{itemize}
	\item Gestión de cuentas para el profesorado, contando con la posibilidad de realizar un registro, y posteriormente un login con la cuenta creada.
	\item El profesorado debe contar con una cuenta válida de GitHub, cuyo nombre de usuario debe ser el mismo que el creado en la cuenta para el servicio implementado.
	\item Cada tutor debe contar con un token de GitHub que se introducirá junto con el registro, de tal forma que el software pueda acceder a los repositorios privados del tutor.
	\item El software debe facilitar la gestión de asignaturas, donde los tutores puedan representar sus asignaturas, utilizando una cadena de coincidencia, que los alumnos introducirán en en nombre del repositorio, de tal forma que se realice el filtrado de los repositorios de cada asignatura.
	\item Se debe ofrecer la posibilidad de añadir un repositorio concreto en una asignatura, de forma persistente, para aquellos casos en los que la cadena de coincidencia no sea correctamente introducida por los alumnos.
	\item Se ofrece una barra de búsqueda por cadena de caracteres para el filtrado de los repositorios.
	\item Se muestran estadísticas sobre el número de colaboradores del proyecto, y su identificación.
	\item Se muestran estadísticas sobre el número de commits totales realizados por cada contribuyente y la distribución en el curso de los mismos.
	\item Se muestra de forma simple y clara el contenido de los commits (Patch), pudiendo visualizar los cambios realizados sobre cada archivo del repositorio.
 
	
\end{itemize}
% \subsection{Arquitectura del lenguaje}

% Deimos ha sido construido con Xtext~\cite{efftinge_spoenemann}, un framework construido en Java para diseñar lenguajes de programación y lenguajes de dominio específico. 

% Xtext proporciona además herramientas para parsear, linkar, validar y compilar los documentos. Es por esto que nuestro generador utiliza una clase generada por Xtext capaz de leer un documento y devolverlo en forma de objeto para su uso, convirtiendo a los diferentes tipos del lenguaje Java cada uno de los atributos especificados.




%%% Local variables:
%%% TeX-master: "memoria.tex"
%%% coding: utf-8
%%% ispell-local-dictionary: "spanish"
%%% TeX-parse-self: t
%%% TeX-auto-save: t
%%% fill-column: 75
%%% End:
