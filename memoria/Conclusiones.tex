\chapter{Conclusiones y vías futuras\label{05conclusiones}}

\section{Conclusiones}

De acuerdo con lo expuesto anteriormente, al comienzo del trabajo se
planteó el problema que supone la administración y gestión de los
repositorios Git de los alumnos por parte del profesorado de cara a evaluar
el trabajo realizado. Este es un problema común al que se enfrentan muchos
de los docentes de la Facultad de Informática de Murcia, especialmente en
las épocas de evaluación.

Como consecuencia, se planteó la posibilidad de analizar las opciones
existentes para simplificar el problema, encontrando así que las soluciones
existentes no se adaptan del todo para la facultad, bien sea por una
difícil implantación en la institución, o por requerimiento de pago por
parte de los profesores o por la misma institución. De este modo, se
propone implementar una solución desarrollada en la Facultad y para la
misma, que intente resolver el problema de una forma sencilla y funcional,
de forma que la solución obtenida evite los problemas de las soluciones ya
analizadas.

Por otra parte, se pretende ofrecer una implementación de forma que pueda
ser desplegada y escalada de forma simple, pudiendo así desplegarse en los
servidores de la facultad de forma simple, y ofreciendo abiertamente la
herramienta para todos los profesores. Además, se ofrece como software
utilizable por otras instituciones y escalable al Cloud.

En cuanto a la solución obtenida, es una solución válida para el problema,
cumpliendo con la necesidad de hacer más eficiente el análisis de las
estadísticas de repositorios, disminuyendo de forma drástica el tiempo
necesario para ello, respecto a una comprobación de manera manual en los
proveedores de servicios. Por otra parte, la herramienta está plenamente
desarrollada en la facultad, por lo que cuenta con la ventaja de no tener
coste alguno, y la gestión interna de los datos de los profesores tampoco
sería expuesta fuera de la misma. Por lo que conseguimos mejorar algunos de
los aspectos ante las soluciones de pago que previamente habíamos
analizado.

Respecto a la facilidad de despliegue planteada, y la posibilidad de
escalado, se ofrecen contenedores software, en este caso usando la
tecnología Docker, con la cual podemos realizar un despliegue de forma
sencilla y rápida, permitiendo además realizar el despliegue en diferentes
plataformas y sistemas operativos. Por lo cual este segundo objetivo queda
resuelto.

El desarrollo de la aplicación, por otra parte, ha sido un reto, el cual ha
conllevado muchas horas de trabajo y la necesidad de aprender nuevas
tecnologías que nunca había tratado. Sin embargo, ha sido muy gratificante
al ver una vez la herramienta completada y poder ponerla en marcha,
entendiendo así la mejora y la ayuda que esta puede suponer. Por otra
parte, el desarrollo de este proyecto no hubiese sido posible sin la ayuda
del tutor Diego Sevilla, el cual me ha introducido en el uso de las nuevas
tecnologías de contenerización y ayudándome en el estudio de las
herramientas necesarias y ofreciendo una nueva visión en la que las
aplicaciones se desarrollan mediante la implementación de múltiples
microservicios independientes, que colaboran para resolver un propósito
común.


\section{Futuros trabajos}

A medida que el desarrollo del proyecto ha ido avanzando, creciendo y
tomando forma, hemos podido comprobar las múltiples posibilidades que se
ofrecen, y el gigantesco rango de mejora que puede tener la aplicación
implementada. En primer lugar, mediante el API de Github, hemos podido ver
que se puede realizar un control total de los repositorios de Git, con
ello, se nos ofrece la posibilidad de no solo consultar estadísticas, sino
de también poder clonar repositorios, realizar commits, añadir
colaboradores y cualquier funcionalidad posible sobre los repositorios,
cuenta y usuarios de Git. Por otra parte, se podría extender los proveedores de 
servicios soportados, en este caso, se soporta el uso de GitHub, por ser el 
proveedor mas usado actualmente, pero otros como GitLab o BitBucket también
sería interesante contar con un soporte para los mismos. Para ello se podría 
especificar adicionalmente los tokens de los otros proveedores y adaptar las 
peticiones del servidor proxy para ejecutar las llamadas al API correspondiente 
en cada caso.

Respecto al análisis de las herramientas ofrecidas por los proveedores de
servicios, especialmente en el caso de la herramienta de pago GitHub
classrooms, vemos nuevas posibilidades de implementación para nuestra
aplicación, pudiendo implementar sobre nuestra arquitectura la gestión de
aulas, donde se configuren profesores y alumnos, ofreciendo múltiples
funcionalidades nuevas como la creación de tareas, esquemas de código que
se comparten para todos los alumnos a modo de plantillas, tareas que puedan
ser autoevaluadas, y muchas más funcionalidades que se pueden plantear,
siendo estas de gran utilidad para tanto estudiantes como profesores.

Por último, respecto al despliegue sobre los servidores de la universidad,
se ofrece una configuración de Docker Compose donde se despliegan los tres
servicios de la aplicación de forma automática. Sin embargo, al ya contar
con los contenedores software, podríamos desplegar esta aplicación sobre
una arquitectura Cloud real, pudiendo ser sobre los mismos servidores de la
facultad, o sobre Cloud ofrecido por algún proveedor de servicios. De esta
manera, mediante un orquestador, como podría ser el caso de Kubernetes, nos
permitiría realizar el escalado de los contenedores de forma independiente,
permitiéndonos así ofrecer más nodos cuando esperamos una mayor carga de
trabajo o número de peticiones sobre la web, como podrían ser las fechas de
evaluación donde los profesores podrían hacer un uso intensivo de la
aplicación.


%%% Local variables:
%%% TeX-master: "memoria.tex"
%%% coding: utf-8
%%% ispell-local-dictionary: "spanish"
%%% TeX-parse-self: t
%%% TeX-auto-save: t
%%% fill-column: 75
%%% End:
